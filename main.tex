\documentclass{article}
\usepackage[utf8]{inputenc}

\usepackage{vmargin}

\setpapersize{A4}
\setmargins{2.54cm}       
{2.54cm}                        
{15.5cm}                      
{21.42cm}                    
{5pt}                           
{1cm}                           
{5pt}                            
{2cm} 

\title{Hilos}
\date{Julio 2020}

\usepackage{natbib}
\usepackage{graphicx}

\begin{document}


\begin{titlepage}
\centering
{\bfseries\LARGE Universidad de Atioquia\par}
\vspace{1cm}
{\scshape\Large Facultad de Ingenier\'ia Electr\'onica \par}
\vspace{3cm}
{\scshape\Huge Hilos \par}
\vspace{3cm}
{\itshape\Large Proyecto de Investigaci\'on \par}
\vfill
{\Large Autor: \par}
{\Large Daniel Felipe Y\'epez Taimal \par}
\vfill
{\Large Julio 2020 \par}
\end{titlepage}

\maketitle


Los hilos en el ámbito de los microprocesadores los podemos definir como el flujo de instrucciones para realizar una tarea, pero ¿por qué es esto tan importante?, porque es mucho más simple y gasta menos recursos dividir un proceso, que gasta más tiempo y recursos, en pequeñas partes que pueden ser completadas con mayor facilidad por el procesador, lo que se traduce como en una mayor optimización de los recursos. Para poder entender lo que son los hilos y para que sirven, es necesario comprender la diferencia entre estos y los núcleos de un procesador, los segundos los podemos definir como la unidad de control de la CPU, y son capaces de realizar una sola instrucción a la vez, así que si tengo varios núcleos será posible ejecutar tantas instrucciones como núcleos tenga. \\

Aunque los núcleos y los hilos sean diferentes están bastante ligados porque a pesar de que los núcleos pueden realizar una sola instrucción, los hilos nos ayudan a realizar muchas instrucciones casi al mismo tiempo, actualmente para los núcleos es imposible realizar dos instrucciones de forma simultánea, pero los hilos al dividir una gran instrucción en pequeñas partes dan la impresión de poder hacerlo, aunque realmente lo que pasa es que  los núcleos realizan cada pequeña instrucción con una gran velocidad. Por lo general cada núcleo tiene un máximo de 2 hilos, pero se debe aclarar que esto no se traduce como una mayor potencia.\\

Así pues los hilos nos brindan muchas ventajas, entre las más destacables esta la mejora en el tiempo de respuesta, cada proceso que este en los hilos comparten recursos y memoria, además permiten usar múltiples núcleos a la vez, esto se traduce como una mayor optimización de los recursos; a pesar de esto para que se pueda hacer uso de los hilos cada programa tiene que estar hecho para poder para poder hacer uso de los mismos, entonces el programador a nivel de Software tiene que pensar bien en cada programa que realice.\\

Con el concepto claro de los hilos, a nivel de software, ¿Cuál es la mejor manera de utilizarlos?, para eso es importante pesar siempre en cómo nos gustaría que funcionar una aplicación, pero desde el punto de vista del usuario; así pues, pensamos en una aplicación rápida y sin tiempos de carga, esto se puede lograr gracias a los hilos, donde podemos dividir todas las tareas que tenemos. En un primer plano o hilo principal vamos a tener la interfaz del usuario, técnicamente vamos a hablar del frontend, este hilo va a ser donde el usuario siempre va a estar, los programadores deben evitar hacer cálculos complejos en este hilo porque influye directamente en la experiencia del usuario; por otra parte tenemos el segundo plano o hilo secundario, conocido también como backend, aquí se van a desarrollar los otros hilos, el segundo y primer plano se van a desarrollar al mismo tiempo, pero el usuario solo va a ver el primer plano, el segundo se va a encargar de las ejecuciones pesadas del programa\\ 







\end{document}
