\documentclass{article}
\usepackage[utf8]{inputenc}

\usepackage{vmargin}

\setpapersize{A4}
\setmargins{2.54cm}       
{2.54cm}                        
{15.5cm}                      
{21.42cm}                    
{5pt}                           
{1cm}                           
{5pt}                            
{2cm} 

\title{Hilos}
\date{Julio 2020}

\usepackage{natbib}
\usepackage{graphicx}

\begin{document}


\begin{titlepage}
\centering
{\bfseries\LARGE Universidad de Atioquia\par}
\vspace{1cm}
{\scshape\Large Facultad de Ingenier\'ia Electr\'onica \par}
\vspace{3cm}
{\scshape\Huge Hilos \par}
\vspace{3cm}
{\itshape\Large Proyecto de Investigaci\'on \par}
\vfill
{\Large Autor: \par}
{\Large Daniel Felipe Y\'epez Taimal \par}
\vfill
{\Large Julio 2020 \par}
\end{titlepage}

\maketitle


Los hilos en el ámbito de los microprocesadores los podemos definir como el flujo de instrucciones para realizar una tarea, pero ¿por qué es esto tan importante?, porque es mucho más simple y gasta menos recursos dividir un proceso, que gasta más tiempo y recursos, en pequeñas partes que pueden ser completadas con mayor facilidad por el procesador, lo que se traduce como en una mayor optimización de los recursos, para poder entender lo que son los hilos y para que sirven, es necesario comprender la diferencia entre estos y los núcleos de un procesador, los segundos los podemos definir como la unidad de control de la CPU, y son capaces de realizar una sola instrucción a la vez, así que si tengo varios núcleos será posible ejecutar tantas instrucciones como núcleos tenga.\\

Aunque los núcleos y los hilos sean diferentes están bastante ligados porque a pesar de que los núcleos pueden realizar una sola instrucción, los hilos nos ayudan a realizar muchas instrucciones casi al mismo tiempo, actualmente para los núcleos es imposible realizar dos instrucciones de forma simultánea, pero los hilos al dividir una gran instrucción en pequeñas partes dan la impresión de poder hacerlo, aunque realmente lo que pasa es que  los núcleos realizan cada pequeña instrucción con una gran velocidad; por lo general cada núcleo tiene un máximo de 2 hilos, pero se debe aclarar que esto no se traduce en una mayor potencia.\\






\end{document}
