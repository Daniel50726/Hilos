\documentclass{article}
\usepackage[utf8]{inputenc}

\usepackage{vmargin}

\setpapersize{A4}
\setmargins{2.54cm}       
{2.54cm}                        
{15.5cm}                      
{21.42cm}                    
{5pt}                           
{1cm}                           
{5pt}                            
{2cm} 

\title{Hilos}
\date{Julio 2020}

\usepackage{natbib}
\usepackage{graphicx}

\begin{document}


\begin{titlepage}
\centering
{\bfseries\LARGE Universidad de Atioquia\par}
\vspace{1cm}
{\scshape\Large Facultad de Ingenier\'ia Electr\'onica \par}
\vspace{3cm}
{\scshape\Huge Hilos \par}
\vspace{3cm}
{\itshape\Large Proyecto de Investigaci\'on \par}
\vfill
{\Large Autor: \par}
{\Large Daniel Felipe Y\'epez Taimal \par}
\vfill
{\Large Julio 2020 \par}
\end{titlepage}

\maketitle


Los primeros indicios de hilos se remontan a 1965, donde todavía no se les daba ese nombre, se conocían simplemente como procesos, pero no fue sino hasta 1979 cuando se tuvo el primer prototipo, donde Max Smith utilizo múltiples pilas en un solo proceso pensando en hacer múltiples procesos de fondo. Después de esto este método ha ido avanzando a grandes pasos de la mano de la tecnología, pero ¿por qué son los hilos importantes?, porque los hilos nos permiten dividir procesos que gastan más tiempo y recursos, en pequeñas partes que pueden ser completadas con mayor facilidad por el procesador, lo que se traduce como en una mayor optimización de los recursos. Para poder entender lo que son los hilos y para que sirven, es necesario comprender la diferencia entre estos y los núcleos de un procesador, los segundos los podemos definir como la unidad de control de la CPU, y son capaces de realizar una sola instrucción a la vez, así que si tengo varios núcleos será posible ejecutar tantas instrucciones como núcleos tenga. \\

Aunque los núcleos y los hilos sean diferentes están bastante ligados porque a pesar de que los núcleos pueden realizar una sola instrucción, los hilos nos ayudan a realizar muchas instrucciones casi al mismo tiempo, actualmente para los núcleos es imposible realizar dos instrucciones de forma simultánea, pero los hilos al dividir una gran instrucción en pequeñas partes dan la impresión de poder hacerlo, aunque realmente lo que pasa es que  los núcleos invierten una pequeña cantidad de tiempo en cada hilo y realizan cada pequeña parte de la instrucción con una gran velocidad, esto aunque no lo hacen de forma paralela, sí lo hacen de forma concurrente. Por lo general cada núcleo tiene un máximo de 2 hilos, pero se debe aclarar que esto no se traduce como una mayor potencia.\\

Así pues los hilos nos brindan muchas ventajas, entre las más destacables esta la mejora en el tiempo de respuesta, cada proceso que este en los hilos comparten recursos y memoria, además permiten usar múltiples núcleos a la vez, esto se traduce como una mayor optimización de los recursos; a pesar de esto para que se pueda hacer uso de los hilos cada programa tiene que estar hecho para poder para poder hacer uso de los mismos, entonces el programador a nivel de Software tiene que pensar bien en cada programa que realice; además la arquitectura de hardware también juega un papel importante, porque la cantidad de núcleos son directamente proporcionales a la cantidad de hilos que se pueden ejecutar de manera simultánea.\\

Con el concepto claro de los hilos, a nivel de software ¿Cuál es la mejor manera de utilizarlos?, para eso es importante pesar siempre en cómo nos gustaría que funcionar una aplicación, pero desde el punto de vista del usuario; así pues, pensamos en una aplicación rápida y sin tiempos de carga, esto se puede lograr gracias a los hilos, donde podemos dividir todas las tareas que tenemos. En un primer plano o hilo principal vamos a tener la interfaz del usuario, técnicamente vamos a hablar del frontend, este hilo va a ser donde el usuario siempre va a estar, los programadores deben evitar hacer cálculos complejos en este hilo porque influye directamente en la experiencia del usuario; por otra parte tenemos el segundo plano o hilo secundario, conocido también como backend, aquí se van a desarrollar los otros hilos, el segundo y primer plano se van a desarrollar al mismo tiempo, pero el usuario solo va a ver el primer plano, el segundo se va a encargar de las ejecuciones pesadas del programa.
Existen dos formas de implementar los hilos, estas son los hilos a nivel de usuario o ULT, por sus siglas en inglés, y los hilos a nivel de kernel o KLT. En el primer tipo (ULT) toda la gestión la realiza la aplicación sin soporte del sistema operativo, esta forma de tratar a los hilos tiene muchas ventajas, una de las más destacables esta la forma global en la que sirven, ya que son fácilmente implementados en cualquier SO ya que no dependen directamente de este, y su principal desventaja es que en una implementación pura de ULT no es posible aprovechar todos los recursos de los núcleos ya que al final estos los reconocen como un único proceso. En la implementación de tipo KLT todo el trabajo de implementación lo realiza el kernel, lo cual brinda una gran ventaja frente a la implementación del tipo ULT puesto que aprovecha el total de recursos y el tiempo de respuesta es mucho menor.\\

Debido a que en ambos métodos existen ventajas y desventajas se han creado tres métodos para poder implementar estos dos tipos de manejo de hilos, como el Modelo Mx1 (Many to One), donde a múltiples hilos de tipo ULT se le asigna un hilo de tipo KLT, este modelo al funcionar como una serie tiene un inconveniente, porque en el momento en el que un hilo se bloquea, todo el proceso se bloquea; también esta el Modelo 1x1 (One to One), en este modelo a cada hilo de tipo ULT se le asigna uno de tipo KLT, este presenta un inconveniente a nivel del SO, ya que el numero de hilos de tipo KLT es restringido por este; por ultimo este el Modelo MxM (Many to Many), donde a cada proceso se le asignan múltiples hilos de tipo KLT sin importar el número de hilos creados del tipo ULT y no presenta ninguno de los inconvenientes de los anteriores métodos, ya que recoge lo mejor de cada uno.\\

Entonces para finalizar, se puede concluir que los hilos ayudan a los programadores a usar todos los recursos que la CPU le asigna, les motiva a hacer su trabajo de la mejor manera y de esta forma brindarle al usuario la mejor experiencia posible. Los hilos deben estar presentes en cada programa que un programador realice porque al final, la satisfacción del usuario es la razón es la carta de presentación para cualquier programador y el manejo óptimo de los recursos brindados al mismo siempre será agradecido.\\








\newpage

\bibliographystyle{plain}
\begin{thebibliography}{X}
\bibitem{Br} \textsc{Bryan O'Sullivan B.O},
\textit{The History of threads}, Londres, Inglaterra.
\bibitem{Jes} \textsc{Jesus Garcia Garcia J.G.G},
\textit{ Nucleos e Hilos de un procesador}, Madrid, España.
\bibitem{Ram} \textsc{Ramon Invarato Menendez R.I.M},\textit{ Multitarea e Hilos, facil y muchas ventajas}, Madrid, España.
\bibitem{Uni} \textsc{Universidad de la Republica UdeR},\textit{Hilos}, Montevideo, Uruguay.

\end{thebibliography}
\end{document}
